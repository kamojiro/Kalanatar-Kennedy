\documentclass{article}[12pt]
\usepackage{amsmath}
\usepackage{amsbsy} %%iranaippoi%%
\usepackage{amsthm}
\usepackage{amssymb}
\usepackage{amscd}
%%\usepackage{url}
\usepackage[top=25truemm,bottom=25truemm,left=25truemm,right=25truemm]{geometry}
\usepackage[utf8]{inputenc}
\usepackage[all]{xy}
%%\usepackage[dvipdfm,citecolor]{hyperref}

\author{Ryo Ochi}
\title{Notes on discrete groups and Furstenberg boundaries}

\theoremstyle{break}
\newtheorem{theorem}{Thm}[section]
\newtheorem{corollary}{Cor}[theorem]
\newtheorem{definition}{Def}[section]
\newtheorem{proposition}{Prop}[section]
\newtheorem{lemma}{Lem}[section]
\newtheorem{remark}{Rem}[theorem]
\newtheorem{example}{Ex}

\renewcommand{\<}{\langle}
\renewcommand{\>}{\rangle}
\newcommand{\C}{\mathbb{C}}
\newcommand{\R}{\mathbb{R}}
\newcommand{\N}{\mathbb{N}}
\newcommand{\Z}{\mathbb{Z}}
\newcommand{\Q}{\mathbb{Q}}
\newcommand{\A}{\mathcal{A}}
\newcommand{\T}{\mathfrak{T}}
\newcommand{\D}{\mathcal{D}}
\newcommand{\fN}{\mathfrak{N}}
\newcommand{\fM}{\mathfrak{M}}
\newcommand{\Tr}{\rm{Tr}}
\renewcommand{\O}{\mathcal{O}}
\newcommand{\F}{\mathbb{F}}
\renewcommand{\P}{\mathcal{P}}
\newcommand{\id}{{\rm id}}
\newcommand{\Axis}{{\rm Axis}}
\newcommand{\Aut}{{\rm Aut}}
\newcommand{\Fix}{{\rm Fix}}
\renewcommand{\d}{\frak{\partial}}
\renewcommand{\L}{\rm{L}}
\newcommand{\ind}{\rm{ind}}
\newcommand{\e}{\varepsilon}

\begin{document}
\maketitle
\tableofcontents
%%%
Let $G$ be  a (countable) discrete group.
\section{Introduction}
We will argue C*-simplicity.
So, we recall group C*-algebras.
\begin{definition}
  Let $\lambda$ be a left regular representation of $G$,
  i.e.
  \begin{align*}
    G \ni g \mapsto \lambda_g \, [ = \delta_h \mapsto \delta_{gh} ] \in B(l^2(G)).
  \end{align*}
  We define the reduced group C*-algebra $C_r^*(G)$ by the closure of $ \overline{\rm span}\{\lambda_g\}_{g \in G}$.

  For any descrete group $G$, the reduced C*-algebra $C_r^*(G)$ has a canonical tracial state $\tau_0 := \< \dot \delta_e, \delta_e \>$.
\end{definition}

\begin{definition}
  A group $G$ is called C*-simple, if $C_r^*(G)$ is simple,
  i.e.
  $C_r^*(G)$ has no closed two-sided ideal.

  A group $G$ has the unique trace property if $C_r^*(G)$ has a unique trace.
\end{definition}

Powers \cite{powers1975simplicity} proved that free groups is C*-simple.
\begin{theorem}[\cite{powers1975simplicity}]
  Let $\F_2$ be a free group of rank $2$.
  For any $a \in C_r^*(G)$ and any $ \varepsilon > 0$,
  there are $g_i, \ldots, g_n \in G$ and a partition of unity $\sum_{i=1}^n c_i = 1$ s.t.
  \begin{align*}
    \| \tau_0(a) - \sum_{i=1}^n c_i \lambda_{g_i} a \lambda_{g_i}^* \| < \varepsilon
  \end{align*}
  In particular, $C_r^*(\F_2)$ is simple and has a unique trace.
  In other words, $\F_2$ is C*-simple and has the unique trace property.
\end{theorem}

Variants of Powers' proof became the main method for establishing these property.
Many many results.

Kalantar and Kennedy \cite{Kalantar2017boundaries} made the new method for proving C*-simplicity.
\begin{theorem}[\cite{Kalantar2017boundaries}]
  Let $G$ be a discrete group and $\d_FG$ be the Furstenberg boundary of $G$.
  The followings are equivalent.
  \begin{enumerate}
  \item $G$ is C*-simple;
  \item $C(\d_FG) \rtimes_r G$ is simple;
  \item $C(B) \rtimes_r G$ is simple for some $G$-boundary $B$;
  \item $G$ acts on $\d_FG$ topologically freely;
    \item $G$ acts on $B$ topologically freely for some $G$-boundary $B$:
  \end{enumerate}
\end{theorem}

\begin{remark}[\cite{breuillard2017c}]
  $G$ acts on $\d_F G$ freely if and only if $G$ acts on $B$ topologically freely for some $G$-boundary $B$.
\end{remark}

So, in order to prove C*-simplicity, it suffices to find the boundary on which $G$ acts topologically freely.
For example, for amalgamated free products or HNN-extension, we probably consider the ideal boundary of its Bass-Serre tree.
So, many existing results are proved more simplily.

\section{G-boundary}

\begin{definition}
  Let $G$ be a group and $X$ be a locally compact group on which $G$ acts.
  The action of $G$ is called minimal if $X$ has no non-trivial $G$-invariant closed subspace,
  that is for any $x \in X$, $\overline{G\cdot x} = X$.
  The action of $G$ is called strongly proximal if for any $ \mu \in \P(X)$, $\overline{G\cdot \mu}$ has a dirac mass,
  that is for any $ \mu \in \P(X)$, there exist $x \in X$ and a net $ g_i \in G$ s.t.
  for any $f \in C_0(X)$,
  \begin{align*}
    g_i.\mu(f) = \mu(g_i^{-1}.f) \rightarrow f(x).
  \end{align*}
  $X$ is called G-boundary if $X$ is compact and the action of $G$ is minimal and strongly proximal.
\end{definition}

\begin{example}[$\F_2$]
  We consider a Caley graph of $\F_2$.
  Let $a, b$ be generators of $\F_2$.
  Let $V(T)$ be a set of all words of $\F_2$ and $E(T) = \{ (v,w) \in V(T)\times V(T) |$ there exists $ x \in \{a, a^{-1}, b, b^{-1}\}$ s.t. $v = wx \}$.
\end{example}
\bibliographystyle{alpha}
\bibliography{santorini}
\end{document}
